\documentclass[12pt]{scrartcl}

\usepackage[a4paper,left=3.0cm,right=2.5cm,top=3.0cm,bottom=2.5cm,headheight=15pt,headsep=1.5cm,footskip=1cm]{geometry}


%---- Sonderzeichen-------%
\usepackage {ngerman}



%---- Codierung----%
\usepackage[utf8]{inputenc}	% for Unix and Windows
%\usepackage[applemac]{inputenc}	% for MAC


%----- Mathematischer Zeichenvorrat---%
\usepackage{amsmath}
\usepackage{amssymb}

\usepackage{enumerate}

%----- Mehrseitige Tabellen ----%
\usepackage{longtable}

% fuer die aktuelle Zeit
\usepackage{scrtime}


\usepackage[colorlinks=true, pdfstartview=FitV, linkcolor=black,
citecolor=black, urlcolor=black]{hyperref}


\usepackage{fancyhdr}
\lhead{\nouppercase{\leftmark}} \chead{} \rhead{\thepage}
\fancyfoot {}


\usepackage[normalem]{ulem}
\newcommand{\markup}[1]{\uline{#1}}

\begin{document}



\begin{center}
\textbf{Seminar: "`Software Engineering"', Semester WS 14/15}

Prof. Dr.-Ing. Samuel Kounev, Lehrstuhl für Informatik II, Software Engineering, Institut für Informatik, Universität Würzburg
\end{center}


\subsection*{General data}

\paragraph{Name of the reviewer: Ștefan-Octavian Radu}

\paragraph{Title of the peer-reviewed article:} Function approximation and evaluation methods of mathematical functions

\paragraph{Name of the author of the peer-reviewed article:} Nishant Ingle

\paragraph{Date:} \today

\subsection*{Summary of the peer-reviewed text} The article highlights
implementation for some approximate algorithms for a number of important
mathematical formulas.


\subsection*{Rating}
The rating is given in the table according to the following scale. This scale is based on the typical English review classification: 1 = accepted, I would argue for it (strong accept and I would argue for it); 2 = tend to accept (weak accept); 3 = tend to reject (weak reject); 4 = rejected, I would argue for it (strong reject and I would argue for it).

\begin{longtable}{|p{12cm}|p{2.5cm}|}	
		\hline
			\textbf{Criteria} & \textbf{Rating (number)} \\
		\hline
		\hline
		\endhead
			\textbf{Abstract}		
						
				Is the choice of words in the abstract appropriate? Is the level of abstraction appropriate?
				
                Does the summary clearly indicate what to expect in the article? 
				
                Is the summary understandable without having read the article?			
			& 1 \ \\
		\hline
			\textbf{Formal aspects}	
				
				Does the document have the required length and formatting?
				
				Does the document contain spelling or grammatical errors?
			& 3 \ \\
		\hline	
			\textbf{Outline / Structuring}		
			
				Does the introduction explain the outline?
				
				Are the word choices the level of abstraction appropriate for headings?
				
				Are terms explained before they are used?
				
				Are headings and consistent with section content?
			& 2 \ \\
		\hline	
			\textbf{Writing style and readability}	
			
				How well does the article use technical terms, equations, pseudocode, figures, and tables?

				Is the argumentation clear, comprehensible, and understandable?

				Is the text too prosaic, unscientific, or overly scientific?
				
				Does the comprehensibility suffer from a pseudo"=scientific style?

				Are sentences and paragraphs of an appropriate length?
			& 3 \ \\			
		\hline	
			\textbf{Reference use}
			
				Are references provided when they are necessary?
				
				Are the references given relevant?
				
				Are verbatim text passages marked as citations?
			& 4 \ \\				
		\hline		
			\textbf{Bibliography}
			
				Are all entries of the bibliography complete and correct?
				
				Would references be mentioned that are missing in the article?
				
				Are all critical statements supported by a "literature" reference?
			& 4 \ \\				
		\hline
			\textbf{Level of detail}
			
				Is the presentation of background""=concepts appropriate for the intended readers?
				
				Does the paper maintain an appropriate balance between technical details and "`high"=level"'"=concepts? (Expected to be an overview article).
				
				Are the breadth and level of detail appropriate for an article of this length?
			& 4 \ \\				
		\hline
			\textbf{Completeness}
			
				Is the text complete?			
			& 4 \ \\	
		\hline
			\textbf{Conclusion / Bottom line}
			
				  Is the content of the article adequately summarized?
				
				Is the content presented reflected (critically) in the concluding part?			
			& 4 \ \\				
		\hline
			\textbf{Overall rating}
			& 4 \ \\				
		\hline		
		
\end{longtable}



\subsection*{Explanations}

%Is the choice of words in the abstract appropriate? Is the level of abstraction appropriate?

%Does the summary clearly indicate what to expect in the article? 

%Is the summary understandable without having read the article?			
\paragraph{Abstract:} The abstract is clear, setting clear expectations from the rest of the article.

%----------------------------------------------------

%Does the document have the required length and formatting?

%Does the document contain spelling or grammatical errors?
            
\paragraph{Formal aspects} The document does not contain any apparent spelling
or grammatical errors. However, it is not properly formatted and doesn't contain
the expected amount of content. In fact, the abstract, introduction and
conclusion together contain more content the rest of the paper. The main
section of the paper is named ``Content'', which, besides spatial separation,
doesn't help the reader in any meaningful way. As such, it could be omitted,
and the subsection names, which are more suggestive, could be used as sections.
The main section is merely an enumeration of mathematical functions and
pseudo-code blocks, which would preferably be placed in an appendix and
referenced throughout the article when needed.


%Does the introduction explain the outline?

%Are the word choices the level of abstraction appropriate for headings?

%Are terms explained before they are used?

%Are headings and consistent with section content?

\paragraph{Outline / Structuring} The outline is clearly explained in the
introduction. The words choices for the headings are rather abstract, some of
them being just the mathematical notation of a concept (e.g. $log(x)$ instead
of ``Logarithm'', or $\sqrt{x}$ instead of ``The Square Root''). There are some
brief definitions for some concepts, while others are introduced with no
explanation (e.g. the Newton Raphson). The headings are consistent with the
content, but often vague (e.g. ``Content'', or ``Explanation''). The document
structure could be improved, but it not entirely inappropriate for the given
content. Every newly introduced concept should be accompanied by an explanation,
according to it's importance in the text.
   
%----------------------------------------------------

%How well does the article use technical terms, equations, pseudocode, figures, and tables?

%Is the argumentation clear, comprehensible, and understandable?

%Is the text too prosaic, unscientific, or overly scientific?

%Does the comprehensibility suffer from a pseudo"=scientific style?

%Are sentences and paragraphs of an appropriate length?

\paragraph{Writing style and readability:} The writing style of the author is
more prosaic than scientific, utilising uncommon words or stylistic
constructions that hinder comprehension (e.g. ``elusive'', ``swift and accurate
results''). Moreover, a number of hyperbolic expressions that are present especially 
in the conclusion. Besides the abstract, introduction and conclusion, most of the
paragraphs are one-liners. Mathematical symbols are sometimes used instead of a
word to denote particular concepts (e.g. $\%$ instead of ``percentage''). This
hinders comprehension and makes the paper appear less scientific.

%----------------------------------------------------

%Are references provided when they are necessary?

%Are the references given relevant?

%Are verbatim text passages marked as citations?

%Are all entries of the bibliography complete and correct?

%Would references be mentioned that are missing in the article?

%Are all critical statements supported by a "literature" reference?

\paragraph{Reference use \& Bibliography:} There are no references used
throughout the paper. The author introduced as such mathematical functions, core
statements about them and pseudo-code fragments which stand for computer
implementations of such functions, without mentioning the source. Given the
context, this can be interpreted as plagiarism and not the author's own work.
There are some bibliographic entries at the end of the article, but are not 
mentioned anywhere in the article.

%----------------------------------------------------

%Is the presentation of background""=concepts appropriate for the intended readers?

%Does the paper maintain an appropriate balance between technical details and "`high"=level"'"=concepts? (Expected to be an overview article).

%Are the breadth and level of detail appropriate for an article of this length?

\paragraph{Level of detail / Completeness} Throughout the article, the
mathematical formulas and computer implementation are just stated as such,
without further explanations. This limits very much the level of understanding
that a reader can achieve from this paper. Furthermore, the ``evaluation''
aspect in the title of the paper doesn't seem to be covered, and it's thus
unclear what it refers to. Giving a proper description and background
information on each function presented, and also the reasoning behind the
presented algorithms would be very useful for the reader. Because of the afore
mentioned points, the text cannot be considered complete.

%Is the content of the article adequately summarized?

%Is the content presented reflected (critically) in the concluding part?			

\paragraph{Conclusion / Bottom line:} The conclusion is well written if judged
in isolation. However, given the context, it does not accurately reflect the
content of the paper. For example, it is stated that the paper provides a
comprehensive exploration of the topic at hand, which is not accurate given the
lack of explanations and reasoning. A number of such hyperbolic expressions are
present throughout the conclusion, which are not well suited for a scientific
paper. The conclusion should be rewritten in such a way that it gives a brief
recap of the core aspects covered in the paper, clearly stating the results and
the contribution of the author. A more neuter tone would be preferable, such that
reader can objectively judge the usefulness of the work and not be influenced
by the information's delivery.

%----------------------------------------------------

\paragraph{Strongest aspect(s) of the paper:} The strongest aspect of the paper
is the abstract which does a good job of setting the expectations for the
paper.


\paragraph{Weakest aspect(s) of the paper:} The weakest aspects of the paper are 
the lack of content explaining the stated formulas and algorithms.

\end{document}
