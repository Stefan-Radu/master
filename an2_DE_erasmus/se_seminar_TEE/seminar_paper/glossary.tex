\newcommand{\newglossaryentrywithacronym}[3]{
    % From https://tex.stackexchange.com/questions/8946/how-to-combine-acronym-and-glossary
    %%% The glossary entry the acronym links to   
    \newglossaryentry{#1_gls}{
        name={#1},
        long={#2},
        description={#3}
    }

    % Acronym pointing to glossary
    \newglossaryentry{#1}{
        type=\acronymtype,
        name={#1},
        description={#2},
        first={#2 (#1)\glsadd{#1_gls}},
        see={[Glossary:]{#1_gls}},
    }
}

\newcommand{\newacro}[2]{
    \newglossaryentry{#1}{
        type=\acronymtype,
        name={#1},
        description={#2},
        first={#2 (#1)},
    }
}

\newglossaryentrywithacronym{TEE} {Trusted Execution Environments} {
    A Trusted execution environment is a feature of modern microprocessors
    which provides a hardware-isolated ``secure area that runs in parallel with,
    but separate, from the normal execution environment'' \cite{tee_ieee_standard}.
    This environment aims to guarantee that the executed code, the runtime states
    and the stored data are integrity and privacy protected. It's end goal is to
    improve the overall security of the whole system, by enabling the safe
    execution of security-crytical applications (\cite{tee_app_rev},
    \cite{tee_in_securities}, \cite{tee_is_and_not})
}

\newglossaryentrywithacronym{TA}{Trusted Applications} {
    The code that is executed inside a TEE is commonly referred to as a
    \emph{Trusted Application}, although it typically is just a small portion of a
    larger application that handles sensitive data \cite{tee_app_rev}
}

\newglossaryentrywithacronym{TCB}{Trusted Computing Base}{
    ``The hardware and software components that are used to achieve the security
    protections of a TEE is called the Trusted Computing Base (TCB)''
    \cite{tee_hw_sup}. Most often, it is desirable to minimize the TCB, thus
    reducing the possible number of bugs and vulnerabilities and making it more
    suitable to be formally verified. However, many solutions choose to have
    mutable components as part of the TCB, which enables updating it down the line,
    instead of relying on formal methods to guarantee safety. This will be further
    highlighted when discussing the security principles of the most common TEE
    implementations
}

\newglossaryentrywithacronym{SW} {Secure World} {
    When discussing TEE it is important to distinguish between code execution
    inside and outside of an enclave. We will name the inside of an enclave the
    ``Secure World''
}

\newglossaryentrywithacronym{NW}{Normal World}{
    When discussing TEE it is important to distinguish between code execution
    inside and outside of an enclave. We will name the outside of an enclave
    the ``Normal World''. Any type of execution in the NW will be named as
    untrusted execution
}

\newglossaryentry{enclave} {
    name={Enclave},
    description={
        Even thought \emph{enclaves} are often confused or used interchangeably with
        TEE, those terms are subtly different. As defined in the \emph{IEEE standard}
        \cite{tee_ieee_standard}, an enclave is a protected memory area, the contents
        of which cannot be tampered with, or read from outside the enclave. Most often,
        enclaves are instances of TEE, while TEE are the entire set of technologies
        that enable the creation of enclaves
    }
}

\newglossaryentry{mutable} {
    name=Mutability,
    description={
        In this context, we refer to a component of the TCB as being \emph{mutable}
        when it can be updated post-manufacturing. Mutable components are usually
        implemented in software, but there are exceptions, such as CPU microcode
        \cite{microcode_update}. \emph{Immutable} components cannot be changed after
        manufacturing, are most commonly implemented in hardware and are prevalent as
        \emph{Root of Trust} (see \ref{rot}) (eg. PUF \cite{tee_base_article}).
        Software can be also immutable (eg. Boot ROM as part of Secure Boot)
    }
}

\newglossaryentry{BUS} {
    name=BUS,
    description={
        ``In computer architecture, a bus is a communication system that
        transfers data between components inside a computer, or between
        computers. This expression covers all related hardware components
        (wire, optical fiber, etc.) and software, including communication
        protocols.'' \cite{wiki_bus}
    }
}

\newglossaryentrywithacronym{TPM} {Trusted Platform Module} {
    The Trusted Platform Module (TPM) can either refer to an international standard
    for a microcontroller designed to enable trust in computing platforms by
    securing hardware through integrated cryptographic keys, or to specific chips 
    which follow this standard \cite{iso_11889}, \cite{wikipedia_tpm}
}

\newacro{DMA}{Direct Memory Access}
\newacro{MMIO}{Memory Mapped IO}
\newacro{SDK}{Software Development Kit}
\newacro{TCON}{Trusted Containers}
\newacro{WASM}{Web Assembly}
\newacro{WASI}{WebAssembly System Interface}
\newacro{SCA}{Side-Channel Attacks}
\newacro{FI}{Fault Injection Attacks}
\newacro{TZ}{(ARM) TrustZone}
\newacro{MDS}{Micro-architectural data sampling}
\newacro{RTM}{Root of Trust (Measurement)}
\newacro{SRTM}{Static RTM}
\newacro{DRTM}{Dynamic RTM}
\newacro{MPU}{Memory Protection Units}
\newacro{MMU}{Memory Management Units}
\newacro{ML}{Machine Learning}
\newacro{VM}{Virtual Machines}
\newacro{CPU}{Central Processing Unit}
