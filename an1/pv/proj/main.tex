\documentclass[a4paper,10pt,twocolumn,english]{article}

%---------------------------------------------------------

\usepackage{babel}
\usepackage[utf8]{inputenc}
\usepackage[T1]{fontenc}
\usepackage[hidelinks]{hyperref}
\usepackage{graphicx}
\usepackage{lipsum} 
\usepackage{titlesec}
\usepackage{listings}
\usepackage{color}
\usepackage{abstract}

\definecolor{dkgreen}{rgb}{0,0.6,0}
\definecolor{gray}{rgb}{0.5,0.5,0.5}
\definecolor{mauve}{rgb}{0.58,0,0.82}
\definecolor{somepaleyellow}{rgb}{0.98,0.98,0.93}

\lstdefinestyle{lststyle}{
    backgroundcolor=\color{somepaleyellow},
    numberstyle=\tiny\color{gray},
    keywordstyle=\color{blue},
    commentstyle=\color{dkgreen},
    stringstyle=\color{mauve},
    basicstyle=\ttfamily,
    breakatwhitespace=false,
    breaklines=false, 
    captionpos=b,
    keepspaces=true,
    numbers=left,
    numbersep=5pt,
    showspaces=false,
    showstringspaces=true,
    showtabs=false,
    tabsize=2
}

% Utilizează biblatex pentru referințe bibliografice
\usepackage[
    maxbibnames=50,
    sorting=nty,
    backend=biber,
    style=numeric
]{biblatex}

\addbibresource{bibliography.bib}

\lstset{style=lststyle}

\lstnewenvironment{code}[1][]%
{
   \noindent\minipage{\linewidth}\lstset{#1}}
{\endminipage}

\graphicspath{./images/}
\usepackage[margin=0.5in, bottom=1in]{geometry}

\setlength{\columnsep}{7mm}

%---------------------------------------------------------
% Equation reference, e.g. (1)
\titlespacing*{\section}{0pt}{15pt}{10pt}
\titlespacing*{\subsection}{0pt}{15pt}{10pt}
\newcommand{\eref}[1]{(\ref{#1})}
\newcommand{\lastacc}{[Accessed: \today]} % last accessed quick command
%---------------------------------------------------------

\title{Bughunting in util-linux}
\author{Stefan-Octavian Radu}
\date{\footnotesize\today}

\begin{document}

\maketitle

\section{Introduction}
      In this report I present the process I went through to find various bugs in the \lstinline{util-linux} collection of Linux utilities. I start by giving a bit of information about the tools used. After that, I describe the runtime analysis process using the \lstinline{valgrind} memory checker and describe my findings. I finish the section with a solution to the identified issue. I go on to explain the static analysis process using the \lstinline{cppcheck} tool. I give examples of many false positive findings, but also of a positively identified bug. In the end I present the conclusions of the experiment.

\section{Preliminaries}

For this analysis I used the memory checker tool that is bundled with \lstinline{valgrind} \cite{valgrind} for runtime analysis and the \lstinline{cppcheck} \cite{cppcheck} tool for static analysis.

\subsection{Valgrind}

Valgrind is an instrumentation framework for building dynamic analysis tools. There are also standard tools provided as part of Valgrind itself, which can be used for instrumenting programs. When using Valgrind for instrumentation, it runs the binary of the program in an sandboxed environment, on a virtual synthetic CPU. This allows the Valgrind core to pass the code to the specific tool that is in use. Memcheck, which is the specific tool referred to in this report, inserts code which checks every memory allocation and release made during the execution of the program. By tracking memory flow in this manner it can determine various types of memory related issues, such as memory leaks, or unreleased memory. It also provides detailed stack traces which aid in pinpointing the exact source of each problem. \cite{valgrindcore}

To use valgrind with the Memcheck tool, you just have to call the program and provide as an argument the path to your executable: \lstinline{valgrind ./<binary>}. To get a more comprehensive result I also used two extra flags: \lstinline{valgrind --leak-check=full --show-leak-kinds=all}

\subsection{CppCheck}

CppCheck is a static analysis tool designed for C/C++ programs. It claims to be
different from other similar tools because it uses unsound flow sensitive
analysis \cite{cppcheck}, as opposed to path sensitive analysis which is used
by most other tools. At a high level, flow sensitive means that the tool will
take into account the order of operations in the program and will compute an
answer for each point in the program. Moreover it is considered unsound,
meaning that it doesn't guarantee that all errors that exist in the program
will be reported, despite the error types being among the ones supported by
CppCheck \cite{soundness}.

To use CppCheck you just have to call the program and provide as an argument the path to the code you want to analyse and, optionally, an extra argument with the
path of any \lstinline{include} directory. In my specific use case I used the following command in the root of the project: \lstinline{cppcheck -I ./include ./}. 

\section{Runtime Analysis}

\subsection{About the target}

My analysis started from the \lstinline{cal} terminal utility. Running the
\lstinline{valgrind} memory checker on the already installed instanced I could
see that a few thousand kilobytes of memory were either lost or not properly
freed upon the program exist. \lstinline{cal} is a simple utility program that
runs in the terminal and displays a calender with the current day highlighted
using the \lstinline{ncurses} library. It is part of a bigger suite of CLI
utilities and programs bundled together in the \lstinline{util-linux}
repository \cite{git}. The repository hosted on Github, which this paper is
based on is used for specifically for development.

\subsection{Investigation process}

The first step was to clone the repository and compile the targeted program.
Running \lstinline{valgrind} on the compiled program leads to the results shown
in Listing \ref{innitial}.

\begin{code}[basicstyle=\ttfamily\small,
    caption=Innitial Leak Summary, label=innitial]
LEAK SUMMARY:
   definitely lost: 0 bytes in 0 blocks
   indirectly lost: 0 bytes in 0 blocks
     possibly lost: 0 bytes in 0 blocks
   still reachable: 17,760 bytes in 20 blocks
        suppressed: 0 bytes in 0 blocks
\end{code}

Seeing \lstinline{still reachable} memory is sign of improper handling and
freeing of allocated memory before a program exits. Using the additional flags
\lstinline{--leak-check=full} and \lstinline{--show-leak-kinds=all} I can get a
stack trace which allows me to track the calls which caused the leak.

Following the stack trace and testing the code along the way I understood that
while initializing the necessary elements needed for colour support, there is
also a check performed that verifies if the current terminal in use supports
colours.

\begin{code}[language=c, basicstyle=\ttfamily\small,
    caption=Stack Trace]
// cal.c
if (colors_init(ctl.colormode, "cal") == 0)
    ...
// colors.c
if (cc->mode == UL_COLORMODE_UNDEF
    && (ready = colors_terminal_is_ready()))
    ...
// colors.c
if (setupterm(NULL, STDOUT_FILENO, &ret) == 0
        && ret == 1)
\end{code}

The \lstinline{setupterm} function is part of the \lstinline{ncurses} system
library. While consulting its manual \cite{ncursesmanual} I learned that, as
the name suggests, \lstinline{setupterm} is a routine that handles
initialization of various low-level terminal-dependant structures and
variables. Upon initialization, the \lstinline{cur_term} global is set to point
to the newly initialized memory segment. Releasing this memory however, is the
responsibility of the developer who should call \lstinline{del_curterm}. A
careful inspection of the code shows that \lstinline{del_curterm} is not called
while using the \lstinline{cal} program.

\subsection{Solution}

The solution I came up with was to ensure that when \lstinline{setupterm} is
called, \lstinline{del_curterm} will also be called. For this I used
\lstinline{atexit} from the standard C library which calls a provided function
when the program exits. I thus created a wrapper around the
\lstinline{del_curterm} function and pass it as an argument to the
\lstinline{atexit} call as seen in Listing \ref{atexit}.

\begin{code}[language=c, basicstyle=\ttfamily\small,
    caption=Wrapper for atexit, label=atexit]
/* atexit() wrapper */
static void colors_del_curterm(void)
{
	del_curterm(cur_term);
}
...

if (setupterm(NULL, STDOUT_FILENO, &ret) == 0
        && ret == 1) {
    ...
    atexit(colors_del_curterm);
}
\end{code}

\subsection{Results}

Following the proposed change, I recorded its effects by running
\lstinline{valgrind} again on the newly compiled binary. As shown in Listing
\ref{final}, I managed to reduce the improperly released memory by 9640 bytes
in a total of 15 blocks.

\begin{code}[basicstyle=\ttfamily\small,
    caption=Leak Summary after changes, label=final]
LEAK SUMMARY:
   definitely lost: 0 bytes in 0 blocks
   indirectly lost: 0 bytes in 0 blocks
     possibly lost: 0 bytes in 0 blocks
   still reachable: 8,120 bytes in 5 blocks
        suppressed: 0 bytes in 0 blocks
\end{code}

Since the problem originated from the \lstinline{colors.h} library, this change
affects not only the \lstinline{cal} utility, but also any other program from the
repository which imports \lstinline{colors.h}. This includes common utilities
such as \lstinline{fdisk}, or \lstinline{hexdump}.

\subsection{Further efforts}

As it's obvious from Listing \ref{final}, there is still unreleased memory when
the process finishes execution. By following the cues from \lstinline{valgrind}
I concluded the remaining problems originate in the
\lstinline{tigetnum("colors");} as well as in low level code in the
\lstinline{ncurses} library. After unsuccessful attempts of modifying to code
from the \lstinline{ncurses} library in a meaningful way and some more research
on the topic, I came across an interesting finding in the ncurses FAQ page
\cite{ncursesfaq}. It seems that reports regarding memory still in use in
programs which depend on the \lstinline{ncurses} library are normal and
expected. There are certain chunks of memory which are never freed for
performance reasons.

The final verdict regarding the 5 blocks of memory still in use is thus
inconclusive. There is a high probability that not properly releasing the
respective blocks is intentional, but I couldn't find clear evidence for this.

\section{Static Analysis}

\subsection{About the target}

I thought the best course of action for the Static Analysis part was to verify as much code as possible. As such, I decided to use the CppCheck program on all the code that was available in the repository, thus checking for bugs in every available tool.

\subsection{False Positives}

Calling \lstinline{cppcheck} in the root of the project and filtering for
errors outputs more than 400 issues with the codebase. On a closer inspection I
found the following:

\begin{itemize}
\setlength\itemsep{0em}
\setlength\parsep{0em}
    \item The majority of issues reported were ``null pointer dereference''
        errors. Most of them were reported in scenarios where a function was
        called with a `struct type' as a parameter. All turned out to be false
        positives. 
    \item There were a few reports of ``uninitialized variable''. Analysing the
        specific scenarios showed that the respective variables were
        initialized in some initialization function to which the corresponding
        address was passed. False positive.
    \item There was one report of ``double free''. In that particular case the
        freed pointer was either reallocated or set to \lstinline{NULL}, which
        wouldn't cause any issues. False positive.
    \item There was one report of ``index out bounds access''. In that
        particular case, there was an explicit bound-check wrapping the memory
        access. False positive.
    \item There were a few ``integer overflow'' errors reported. All of the
        cases involved shifting to the left the number 1 with a value
        (\lstinline{1<<K}), which turned out to be less than or equal to 31.
        This wouldn't cause an integer overflow. False positive.
    \item There was a very interesting ``null pointer dereference'' report
        which I emphasized in Listing \ref{pointerderef}. This is actually a
        declaration using the \lstinline{__typeof__} construct called on a null
        pointer variable. This however is not incorrect. False positive.

\end{itemize}

\begin{code}[basicstyle=\ttfamily\small, language=c,
    caption=Declaration using \_\_typeof\_\_, label=pointerderef]
    __typeof__(ask->data.num) *num;
\end{code}

\subsection{One identified bug}

Out of the more than 400 reports, more than $99\%$ of them were false
positives. However, there was one report of ``path with no return statement''.
Analysing the source code led me to conclude that a bug is indeed present
there. Looking at Listing \ref{noreturn} we can see that there are 3 return
paths handled but 4 present. If the conditional evaluates to false and none of
the pre-processor checks hold, then the function will end without returning.
This leads to undefined behaviour, which must be avoided. This is especially
severe as it occurs in a function which deals with authentication related
issues.

To solve this issue, an appropriate return statement should be appeneded to the
function.

\begin{code}[basicstyle=\ttfamily\small, language=c,
    caption=Path with no return, label=noreturn]
...
	if (su->suppress_pam_info
    && num_msg == 1
    && msg
    && msg[0]->msg_style == PAM_TEXT_INFO)
        return PAM_SUCCESS;

#ifdef HAVE_SECURITY_PAM_MISC_H
        return misc_conv(num_msg,
            msg, resp, data);
#elif defined(HAVE_SECURITY_OPENPAM_H)
        return openpam_ttyconv(num_msg,
            msg, resp, data);
#endif
    
    /* return expected */
}
\end{code}

CppCheck claims to report very few false positives. It is very surprising that
despite this claim, more than $99\%$ of my finding were false positives. I
attribute this result to the surprising differences that can occur between the
sytax of a piece of code and its semantics.

\section{Conclusion}

In this report I presented my efforts of finding bugs in the
\lstinline{util-linux} collection of Linux utilities. With the use of the
\lstinline{valgrind} memory checker I managed to identify and also fix an
improper handling of initialized data upon the program exit. This previously
resulted in a certain amount of memory not being released upon the program exit.
Furthermore, with the use of the \lstinline{cppcheck} static analysis tool, I
managed to find a non-void function which has an execution path that doesn't
end with a return statement. This can lead to undefined behaviour. This
experiments prove that the use of both runtime and static analysis tools is
invaluable during development. This is especially relevant when programming
languages such as C, which do not have any memory safety guarantees.
\printbibliography[heading=bibintoc]

\end{document}
